%%% (u)pLaTeX %%%
\RequirePackage{plautopatch} 
\documentclass[dvipdfmx]{jsarticle} 
\usepackage[T1]{fontenc}
\usepackage{xcolor}

%---引用形式の設定---%
\usepackage{cite}
\makeatletter
\def\@cite#1{\hspace*{.15em}\textsuperscript{[#1]}}	%引用を[]で囲み、上付きにする
\makeatother

%%%引用を上付きにしたくないとき(\Cite)
\makeatletter
\def\Cite{\@ifnextchar[% ]
	{\@tempswatrue\let\@cite\@Cite\@citex}
	{\@tempswafalse\let\@cite\@Cite\@citex[]}}
\def\@Cite#1#2{\leavevmode %% \unskip
	\ifnum\lastpenalty=\z@\penalty\@highpenalty\fi% 
	[{\multiply\@highpenalty 3 #1% % 
		\if@tempswa,\penalty\@highpenalty\ #2\fi % 
	}]\spacefactor\@m}
\makeatother
%%% ここまで

\usepackage{hyperref}
\hypersetup{
	dvipdfmx,%
	bookmarks=true,%
	bookmarksnumbered=true,%
	bookmarksdepth=3,%
	colorlinks=true,%
	linkcolor=black,%
	setpagesize=false,%
	citecolor=red,%
	filecolor=black,%
	urlcolor=blue
	}
\urlstyle{rm}
\usepackage{pxjahyper}

\begin{document}
	\BibTeX のテストです\cite{YBCO_PhysRevLett.58.908}。
	
	使い方は同フォルダ内の「*.bst」や「*.bib」を参照してください。
	
	citeパッケージを使うことで引用文献を上付きカギカッコにしています。複数の文献を引用するときの書式なども自由に設定できます。
	\begin{itemize}\setlength{\itemsep}{3pt}
		\item 2つ引用する場合はカンマ区切りになります\cite{YBCO_PhysRevLett.58.908, Kittel_BA74778859}。
		\item 3つ以上引用する場合はハイフンになります\cite{YBCO_PhysRevLett.58.908, Kittel_BA74778859, website1}。
	\end{itemize}
	
	大文字にすると通常サイズになるように設定することも可能です\Cite{YBCO_PhysRevLett.58.908}。	
	プリアンブルの「引用を上付きにしたくないとき」以下をコピペしてご利用ください。
	
	不具合や改善点があれば教えてください。
	
	
	\clearpage
	\bibliographystyle{REVTeX-ja}   % *.bstファイルの名前
	\bibliography{ref}  % *.bibファイルの名前
\end{document}